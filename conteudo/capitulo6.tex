%---------- Sexto Capitulo ----------
\chapter{Conclusões e Trabalhos Futuros}\label{cha:conclusao}

\section{Conclusões}
Os dispositivos móveis devem ser vistos como ferramentas que auxiliam e facilitam a vida do usuário. Como facilitadores do dia-a-dia, estas ferramentas devem possuir aplicativos que evitem expor as informações do usuário a riscos desnecessários e que apresente ao usuário serviços que realmente sejam de utilidade.

Cada tipo de serviço apresenta um conjunto de atributos que são relevantes para o usuário no momento da escolha do serviço e conhecendo previamente quais são os valores de cada atributo, para cada usuário, essa escolha pode ser automatizada.

Este foi o objetivo deste trabalho de conclusão de curso, a implementação de aplicações que permitam a configuração desses atributos relevantes ao usuário e a escolha automática do serviço baseando-se nas configurações definidas.

Assim foi necessário a implementação de uma aplicação para dispositivos móveis onde o usuário configura quais são os valores para os atributos de cada tipo de serviço e que também, além de interface de configuração, é a aplicação consumidora dos serviços. Como também a implementação de uma aplicação servidora que decide qual é o serviço que atende aos requisitos do usuário e realiza as autenticações necessárias de forma transparente para que o usuário tenha o serviço disponibilizado.

Para validar o funcionamento dos sistemas foram implementados serviços que simulam serviços de metereologia de algumas cidades do Paraná e um desses provedores necessita que seja realizada autenticação para o uso do serviço. Devido a implementação do filtro de metereologia o \textit{Broker} realizou as escolhas corretas e fez as autenticações necessárias automaticamente e entregou à aplicação no dispositivo móvel os dados fornecidos pelo provedor de serviço para serem consumidos.

\section{Trabalhos futuros}
As propostas para trabalhos futuros são:
\begin{itemize}
	\item Integração com provedores de serviços reais, como por exemplo o serviço de metereologia \textit{OpenWeatherMap};
	\item A implementação de novos filtros como \textit{plugins} a serem adicionados no \textit{Broker} para os outros serviços;
	\item A implementação de \textit{web services} que podem consumir serviços reais, e que oferecem interfaces, com dados tratados, ao usuário da aplicação para dispositivo móvel desenvolvida.
\end{itemize}